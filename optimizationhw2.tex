\documentclass[12pt]{article}
\setlength\headheight{14.5pt}
\title{Homework}
\author{Frederick Robinson}
\date{18 January 2010}
\usepackage{amsfonts}
\usepackage{fancyhdr}
\usepackage{amsthm}
\pagestyle{fancyplain}

\begin{document}



\lhead{Frederick Robinson}
\rhead{Math 368: Optimization}

   \maketitle

\setcounter{tocdepth}{2} 

\tableofcontents

\section{Book Problems}

\subsection{Problem 4.1}
\subsubsection{Question}
Is Theorem 4.1 valid if $x^* \notin \mathrm{int}\  \mathcal{D}$? If yes, provide a proof. If not, a counterexample.
\subsubsection{Answer}
Recall that Theorem 4.1 states: 

\emph{Suppose $x^* \in \mathrm{int}\ \mathcal{D} \subset \mathbb{R}^n$ is a local maximum of $f$ on $\mathcal{D}$, i.e., there is $r>0$ such that $B(x^*,r) \subset \mathcal{D}$ and $f(x^*) \geq f(x)$  for all $x \in B(x^*,r)$. Suppose also that $f$ is differentiable at $x^*$. Then $Df(x^*)=0$. The same result is true if, instead $x^*$ were a local minimum of $f$ on $\mathcal{D}$}

Theorem 4.1 does not hold if we drop the assumption that $x \in \mathrm{int}\ \mathcal{D}$. For, consider the problem of trying to maximize $f(x)$ subject to $x \in [0,5]= \mathcal{D}$. Clearly $x^*=5$ has the property that $x^* \in \mathcal{D}$ is a local maximum of $f$. However, we know that $\frac{d}{dx} f = 1$ for all $x$ including our choice for $x^*$. So this modification of the theorem does not hold.




\subsection{Problem 4.2}
\subsubsection{Question}
Find all the critical points (i.e., points where $f'(x)=0$) of the function $f:\mathbb{R} \to \mathbb{R}$, defined as $f(x) = x-x^2-x^3$ for $ x \in \mathbb{R}$. Which of these points can you identify as local maxima or minima using the second-order test? Are any of these \emph{global} optima?
\subsubsection{Answer}
The critical points are those where $f'(x)=0$ so we compute
\[f(x) = x-x^2-x^3 \Rightarrow f'(x)=1 -2x -3 x ^2\] 
Hence if 
\[ f'(x)=1 -2x -3 x ^2 = 0 \] 
we have
\[x = \frac{2 \pm \sqrt{4-4(-3)}}{2 (-3) } =  \frac{1 \pm 2}{-3} \]
and the critical points are $x= -1$, $x= 1/3$

Now we apply the second order test. First we compute 
\[ f'(x)=1 -2x -3 x ^2 \Rightarrow f''(x)= -2-6x.\] 
Substituting the critical points which we determined above we see that 
\[f''(-1)=4 \quad f''\left(\frac{1}{3}\right)= -4.\]
So, $x=-1$ is a local minimum, whereas $x=1/3$ is a local maximum. Neither of these are global extrema however as $\lim_{x\to \infty}f(x)= -\infty$ and $\lim_{x\to -\infty}f(x)=\infty$.


\subsection{Problem 4.4 (a,b,g)}
\subsubsection{Question}
Find and classify the critical points (local maximum, local minimum, neither) of each of the following functions. Are any of the local optima also global optima?
\begin{enumerate}
\item $f(x,y)=2x^3+ x y ^2+5x^2+y^2$
\item $f(x,y)=e^{2x} (x+y^2+2y)$
\item $f(x,y)= \frac{x}{1+x^2+y^2}$
\end{enumerate}
\subsubsection{Answer}
\begin{enumerate}
\item To locate the critical points we compute the partial derivatives by
\[\frac{\partial}{\partial x} \left( 2x^3+ x y ^2+5x^2+y^2\right) =  6x^2 +y^2+10 x\]
\[\frac{\partial}{\partial y} \left( 2x^3+ x y ^2+5x^2+y^2\right) =  2xy+2y\]
and find those points at which
\[\frac{\partial}{\partial x} f = \frac{\partial}{\partial y} f = 0.\]
So,
\[0=2xy+2y= 2y(x+1)\Rightarrow y=0 \quad \mathrm{or} \quad x=-1\]
and
\[0=6x^2 +y^2+10 x \Rightarrow -y^2 = x(6x+10)\]
Thus we have 
\emph{Case 1: }$y=0 \Rightarrow x=0 \mathrm{\ or\ } x= -5/3$ and \emph{Case 2: }$x=-1 \Rightarrow y=2 \mathrm{\ or\ } y=-2$


So we conclude that the critical points are $(0,0)$, $(-5/3,0)$, $(-1,2)$, $(-1,-2)$, 

To determine which of these are local extrema we first compute the Hessian matrix as 
\[
\left[
\begin{array}{lr}
\frac{\partial}{\partial x \partial x} f&\frac{\partial}{\partial x \partial y} f\\
\frac{\partial}{\partial y \partial x} f&\frac{\partial}{\partial y \partial y} f
\end{array}\right]
=
\left[
\begin{array}{lr}
12 x +10 & 2y\\
2y & 2x+2
\end{array}\right].
\]

Evaluating this matrix at each critical point we get 
\[(0,0)\to\left|
\begin{array}{lr}
10 & 0\\
0 & 2
\end{array}\right|=20
\quad
(-5/3,0)
\to
\left|
\begin{array}{lr}
-10 & 0\\
0 & -4/3
\end{array}\right|=40/3
\]
\[
(-1,2)
\to
\left|
\begin{array}{lr}
2  & 4\\
4 & 0
\end{array}\right|=-16
\quad
(-1,-2)
\to
\left|
\begin{array}{lr}
2  & -4\\
-4 & 0
\end{array}\right|=-16
\]
So, since the corresponding Hessian matrix is positive definite $(0,0)$ is a local minimum. Similarly, since its corresponding Hessian is negative definite $(-5/3,0)$ is a local maximum. The other two critical points are indefinite, and so they are not local extrema. 

It is easy to check though, that these local extrema are not global extrema for in particular $f(0,0)=0>-125=f(-5,0)$ and $f(-5/3,0)=\frac{125}{27}<18000=f(20,0)$

\item Again we compute the partial derivatives in order to find the critical points
\[\frac{\partial}{\partial x} \left[ e^{2x}(x+y^2+2y) \right]=e^{2 x} \left(1+2 x+4 y+2 y^2\right)\]
\[\frac{\partial}{\partial y} \left[ e^{2x}(x+y^2+2y) \right] =e^{2 x} (2+2 y)\]
Since the critical point are those points with both partial derivatives zero and exponentiation always yields a nonzero result we must have $1+2x+4y+2y^2=0$ and $2+2y=0$. So, $y=-1$ and consequently $0=1+2x-4+2\Rightarrow x=1/2$.

Therefore we conclude that the only critical point is $(1/2,-1)$. To test this point we need to compute the Hessian matrix. We do this as
\[
\left[
\begin{array}{lr}
\frac{\partial}{\partial x \partial x} f&\frac{\partial}{\partial x \partial y} f\\
\frac{\partial}{\partial y \partial x} f&\frac{\partial}{\partial y \partial y} f
\end{array}\right]
=
\left[
\begin{array}{lr}
4 e^{2 x} \left(x+(1+y)^2\right) & 4 e^{2 x} (1+y)\\
4 e^{2 x} (1+y) & 2 e^{2 x}
\end{array}\right].
\]

Evaluating this matrix at the critical point we arrive at 
\[
\left[
\begin{array}{lr}
4 e^{2 (1/2)} \left((1/2)+(1+(-1))^2\right) & 4 e^{2 (1/2)} (1+(-1))\\
4 e^{2 (1/2)} (1+(-1)) & 2 e^{2(1/2)}
\end{array}\right].
\]
\[
=
\left[
\begin{array}{lr}
6 e & 0\\
0& 2 e
\end{array}\right].
\]
and since this matrix is positive definite we may conclude that the point $(1,-1)$ is a local minimum.


Furthermore it is apparent that this is a global minimum since any potential minimum needs to have $y=-1$ (since this of course minimizes $y^2+2y$). Fixing this we evaluate $\lim_{x \to \infty} f$ and $\lim_{x \to -\infty} f$ yielding $\infty$ and $0$. Since both of these exceed the value of the function at $(1/2,-1)$ this point must be a global minimum.

\item Again here we begin by computing the partial derivatives
\[\frac{\partial}{\partial x} \left( \frac{x}{1+x^2+y^2} \right) = \frac{1-x^2+y^2}{\left(1+x^2+y^2\right)^2} \]
\[\frac{\partial}{\partial y} \left( \frac{x}{1+x^2+y^2} \right) = -\frac{2 x y}{\left(1+x^2+y^2\right)^2} .\]
The second expression is zero if either $x$ or $y$ is made zero. If $x$ is zero however the first expression cannot be zero as well. So, the only case to consider is that in which $y=0$. In this case we can make the first expression zero by letting $x=1$.

So, we conclude that $(1,0)$ is a critical point of the given function.

The Hessian Matrix is just
\[\left[ \begin{array} {lr} \frac{\partial }{\partial x\partial x} & \frac{\partial }{\partial x\partial y} \\
\frac{\partial }{\partial y \partial x} &\frac{\partial }{\partial y\partial y}  \end{array} \right] 
=
\left[ \begin{array} {lr}\frac{2 x \left(x^2-3 \left(1+y^2\right)\right)}{\left(1+x^2+y^2\right)^3}& -\frac{2 y \left(1-3 x^2+y^2\right)}{\left(1+x^2+y^2\right)^3} \\
-\frac{2 y \left(1-3 x^2+y^2\right)}{\left(1+x^2+y^2\right)^3}&-\frac{2 \left(x+x^3-3 x y^2\right)}{\left(1+x^2+y^2\right)^3} \end{array} \right] 
\]
so in this case we get
\[
\left[ \begin{array} {lr}\frac{2  \left(1 \right)}{\left(1+1\right)^3}& 0\\
0&-\frac{2 \left(1+1\right)}{\left(1+1\right)^3} \end{array} \right] 
=
\left[ \begin{array} {lr}\frac{2}{2^3}& 0\\
0&-\frac{4}{2^3} \end{array} \right] 
=
\left[ \begin{array} {lr}\frac{1}{2^2}& 0\\
0&-\frac{1}{2} \end{array} \right] 
\]
Since the Hessian is negative definite we conclude that the point ${1,0}$ is a local maximum.

Moreover, since in the limit $|(x,y)| \to \infty$ we have $f(x,y) \to 0$ this local maximum is also the global maximum. For, were there a global maximum that is not also a local maximum it would have to occur in this limit.

That is, suppose that there is some global maximum that does not occur in the limit  $|(x,y)| \to \infty$. Then there is some neighborhood of this maximum on which it is the maximum. But this is just the definition of a local maximum.
\end{enumerate}




\subsection{Problem 4.7}
\subsubsection{Question}
Suppose $f:\mathbb{R} \to \mathbb{R}$ has a local maximum at $x$ that is not a strict local maximum. Does this imply that $f$ is constant in some neighborhood of $x$? Prove your answer or provide a counterexample.


\subsubsection{Answer}

Consider the function 

\[f(x)= \left\{ \begin{array} {lr} \sin{(1/x)} & |x|>0 \\ 1& x = 0 \end{array} \right.\]

Clearly there is a maximum of this function at $x=0$ since $\sin{x}$ is at most $1$. However, this is not a strict local maximum, since in every neighborhood of $0$ there is some point $y$ in the function where $f(y)=1$. (This follows by the density of $\mathbb{Q} \in \mathbb{R}$)

Finally, the function $f$ is not constant valued on any neighborhood of $0$.

Thus, we have constructed a counterexample.

\subsection{Problem 5.1}
\subsubsection{Question}
Find the maximum and minimum of $f(x,y)=x^2-y^2$ on the unit circle $x^2+y^2=1$ using the Lagrange multipliers method. Using the substitution $y^2=1-x^2$, solve the same problem as a single variable unconstrained problem. Do you get the same results? Why or why not?


\subsubsection{Answer}

First we solve this problem using Lagrange's Theorem. We set up the Lagrangian by 
\[L(x,y)=f(x,y)+\lambda (x^2+y^2 -1) = x^2-y^2+\lambda x^2+\lambda y^2-\lambda\]
and then compute the derivatives with respect to $x$, $y$ and $\lambda$, setting them equal to zero to yield the following three equations.
\begin{equation}\label{x}0=2x+2\lambda x = 2x (1+\lambda)\end{equation}
\begin{equation}\label{y}0= -2y+2 \lambda y= -2y(1 - \lambda )\end{equation}
\begin{equation}\label{l} 0=x^2+y^2-1\end{equation}
Observe that equation \ref{x} may only be satisfied if $\lambda = -1$ or $x=0$

\emph{Case 1:} Assume  $\lambda = -1$ . Then, we must have $y=0$ to satisfy equation \ref{y}, and using this information we see that $x=\pm 1$ in equation \ref{l}.

\emph{Case 2:} Assume $x=0$. Then, in order to satisfy equation \ref{l} we must have $y=\pm1$.

We have now identified the critical points as $(0,1)$, $(0,-1)$, $(1,0)$, $(-1,0)$. Evaluating the function at these points we get
\[f((0,1))=-1,\quad f((0,-1))= -1\]
\[f((1,0))=1,\quad  f((-1,0))=1\]

If we repeat the process using  the substitution method we reduce the problem of optimizing $f(x,y)=x^2-y^2$ subject to  $x^2+y^2=1$  to the easier problem of optimizing $f(x)=x^2-(1-x^2)$ without constraint.

So, 
\[f(x)=x^2-(1-x^2)=2x^2-1\]
\[f'(x)=4x\]
and $f'(x)=0 \Rightarrow x=0$

Now, using the substitution equation $y^2=1-x^2$ again we see that $x=0\Rightarrow y=\pm 1$. These are only two of the solutions which we got using the Lagrangian technique however. We have already checked these points, observing earlier that
\[f((0,1))=-1,\quad f((0,-1))= -1\]

So, we see that we have only found the local minima, not the local maxima as we did with the lagrangian. The reason that we failed to find the other points is that, upon making the substitution we failed to constrain the problem correctly. The substitution makes little since for values of $y^2 > 1$ since, after all the unit circle contains no such points. Had we tried to solve the one variable optimization problem restricted to $y \in [-1,1]$ we would have found all of the solutions as we did with the Lagrangian technique.



\subsection{Problem 5.3d}
\subsubsection{Question}
Find the maxima and minima of the following functions subject to the specified constraints:
\[f(x,y,z)=xyz \mathrm{\ subject\ to\ } x+y+z=5 \mathrm{\ and\ } xy+xz+yz=8\]


\subsubsection{Answer}

We set up the Lagrangian as usual by 
\[L(x,y,z,\lambda,\mu)= xyz + \lambda_1(x+y+z-5)+\lambda_2(xy+xz+yz-8)\]
Computing all of the partial derivatives and setting them equal to zero we see that
\begin{equation}\label{x1}\frac{\partial L}{\partial x}=yz+\lambda_1+\lambda_2 y +\lambda_2 z=0\end{equation}
\begin{equation}\label{y1}\frac{\partial L}{\partial y}=xz+\lambda_1+\lambda_2 x+ \lambda_2 z=0\end{equation}
\begin{equation}\label{z1}\frac{\partial L}{\partial z}=xy+\lambda_1 + \lambda_2 x+\lambda_2 y=0\end{equation}
\begin{equation}\label{2}\frac{\partial L}{\partial \lambda_1}=x+y+z-5=0\end{equation}
\begin{equation}\label{1}\frac{\partial L}{\partial \lambda_2}=xy+xz+yz-8=0\end{equation}

Now after some algebra (omitted) we see that the solutions are 
\[
\lambda _1= \frac{16}{9},\lambda _2= -\frac{4}{3},x= \frac{4}{3},y= \frac{4}{3},z= \frac{7}{3}
\]
\[
\lambda _1= \frac{16}{9},\lambda _2= -\frac{4}{3},x= \frac{4}{3},y= \frac{7}{3},z= \frac{4}{3}
\]
\[
\lambda _1= \frac{16}{9},\lambda _2= -\frac{4}{3},x= \frac{7}{3},y= \frac{4}{3},z= \frac{4}{3}
\]
\[
\lambda _1= 4,\lambda _2= -2,x= 1,y= 2,z= 2
\]
\[
\lambda _1= 4,\lambda _2= -2,x= 2,y= 1,z= 2
\]
\[
\lambda _1= 4,\lambda _2= -2,x= 2,y= 2,z= 1
\]
Now we just evaluate $f(x,y,z)=xyz$ at each of these points
\[f(\frac{4}{3},\frac{4}{3}, \frac{7}{3})=\frac{112}{27}
\]
\[f(\frac{4}{3}, \frac{7}{3}, \frac{4}{3})=\frac{112}{27}
\]
\[f(\frac{7}{3}, \frac{4}{3}, \frac{4}{3})=\frac{112}{27}
\]
\[f( 1, 2, 2)=4
\]
\[f( 2, 1, 2)=4
\]
\[f(2,2, 1)=4
\]

So, since $\frac{112}{27}$ is greater than $4$ each of the first three points is a local maximum, and the last three are local minima.

They are also globally maxima subject to the constraints since the constraints define a closed and bounded (and therefore compact) subset of $\mathbb{R}^3$

\section{Supplemental Problems} 

\subsection{Problem 4.1}
\subsubsection{Question}
Find all the critical points and classify them as local maximum, local minimum, or neither for the following functions
\begin{enumerate}
\item $f(x,y,z)=x^4+x^2-6xy+3y^2+z^2$
\item $f(x,y,z)=3x-x^3-2y^2+y^4+z^3-3z$
\end{enumerate}
\subsubsection{Answer}
\begin{enumerate}
\item
We compute the partial derivatives to get
\[\frac{\partial}{\partial x} \left( x^4+x^2-6xy+3y^2+z^2 \right) = 2 \left(x+2 x^3\right)\]
\[\frac{\partial}{\partial y} \left( x^4+x^2-6xy+3y^2+z^2 \right) = 6y\]
\[\frac{\partial}{\partial z} \left( x^4+x^2-6xy+3y^2+z^2 \right) = 2z \]
In order that a point be a critical point we must have each partial derivative zero. The only points which satisfy this criterion have $z=0$ and $y=0$. Finally the points must also have $x+2x^3=0$. The only real $x $ which satisfy this last are $x=0$  . Hence, the only critical point is $(0,0,0)$.

So, in order to figure out whether this is a local minimum, local maximum, or neither we compute the Hessian.

\[
\left[
\begin{array}{lcr}
\frac{\partial}{\partial x \partial x} & \frac{\partial}{\partial x \partial y} & \frac{\partial}{\partial x \partial z} \\
\frac{\partial}{\partial y \partial x} & \frac{\partial}{\partial y \partial y} & \frac{\partial}{\partial y \partial z }  \\
\frac{\partial}{\partial z \partial x } & \frac{\partial}{\partial z \partial y} & \frac{\partial}{\partial z \partial z} \\
\end{array}
\right]
=
\left[
\begin{array}{lcr}
2+12 x^2 & 0 &0\\
0 & 6 & 0  \\
0 & 0 & 2\\
\end{array}
\right]
\]
So, evaluating at our critical point we just get 
\[\left[
\begin{array}{lcr}
2 & 0 &0\\
0 & 6 & 0  \\
0 & 0 & 2\\
\end{array}
\right]
\]
and since this matrix is positive definite we must be dealing with  a local minimum. 
\item
We compute the partial derivatives to get
\[\frac{\partial}{\partial x} \left( 3x-x^3-2y^2+y^4+z^3-3z \right) = 3-3 x^2\]
\[\frac{\partial}{\partial y} \left( 3x-x^3-2y^2+y^4+z^3-3z \right) =4 y \left(-1+y^2\right) \]
\[\frac{\partial}{\partial z} \left( 3x-x^3-2y^2+y^4+z^3-3z \right) = 3 \left(-1+z^2\right)\]
So, to make each of these expressions zero we must have $x= \pm1$, $ z= \pm 1$ and $y=0 $ or $y=\pm 1$. That means that the critical points we have are in particular $(1,-1,1)$, $(1,0,1)$, $(1,1,1)$, $(1,-1,-1)$, $(1,0,-1)$, $(1,1,-1)$, $(-1,-1,1)$, $(-1,0,1)$, $(-1,1,1)$, $(-1,-1,-1)$, $(-1,0,-1)$, $(-1,1,-1)$ 

Next we find the hessian matrix 
\[
\left[
\begin{array}{lcr}
\frac{\partial}{\partial x \partial x} & \frac{\partial}{\partial x \partial y} & \frac{\partial}{\partial x \partial z} \\
\frac{\partial}{\partial y \partial x} & \frac{\partial}{\partial y \partial y} & \frac{\partial}{\partial y \partial z }  \\
\frac{\partial}{\partial z \partial x } & \frac{\partial}{\partial z \partial y} & \frac{\partial}{\partial z \partial z} \\
\end{array}
\right]
=
\left[
\begin{array}{lcr}
-6 x& 0&0\\
0 &-4+12 y^2& 0 \\
0 &0& 6z\\
\end{array}
\right]
\]
So we may now categorize the above critical points as follows
\[
\begin{array}{l r }
(-1,1,1),(-1,-1,1),& \mathrm{Minimum}\\
(1,1,-1), (1,-1,-1) & \mathrm{Maximum}\\
Others & \mathrm{Neither}\\
\end{array}
\]
Since, in order to be positive definite we must have $x<0$, $|y|> \sqrt{1/3}$, and $z>0$. Also, in order to be negative definite we must have $x>0$,  $|y|> \sqrt{1/3}$, $z<0$.

\end{enumerate}




\subsection{Problem 1.1}
\subsubsection{Question}
A firm uses two inputs $q_1$ and $q_2$ to produce a single output $Q$, given by the production function $Q=k q_1^{2/5}q_2^{1/5}$. Let $P$ be the price of the output $Q$, $p_1$ be the price of $q_1$, and $p_2$ be the price or $q_2$. The profit is given b $\pi = P k q_1^{2/5} - p_1q_1-p_2q_2$. The inputs that maximize profit satisfy 
\[0=\frac{2 P k}{5} q_1^{-3/5} q_2^{1/5}-p_1 \quad \mathrm{and}\]
\[0=\frac{P k}{5} q_1^{2/5}q_2^{-4/5}-p_2. \]
\begin{enumerate}
\item Show that these two equations can be used to determine the amounts of inputs $q_1$ and $q_2$ in terms of the price $p_1$, $p_2$, and $P$. Show that the relevant matrix has nonzero determinant.
\item Write the matrix equation for the partial derivatives of $q_1$ and $q_2$ with respect to $p_1$, $p_2$ and $P$ in terms of the variables
\item Solve for the matrix of partial derivatives of $q_1$ and $q_2$ in terms of $p_1$, $p_2$ and $P$.
\end{enumerate}
\subsubsection{Answer}
\begin{enumerate}
\item The two equations determine the amounts of inputs in terms of the prices by the implicit function theorem. The relevant matrix is
\[ \left[ \begin{array}{lr} \frac{-6 P k}{25} q_1^{-8/5} q_2^{1/5} & \frac{2 P k}{25} q_1^{-3/5}q_2^{-4/5}\\  \frac{2 P k}{25} q_1^{-3/5} q_2^{-4/5} & \frac{-4 P k}{25} q_1^{2/5}q_2^{-9/5}\\   \end{array} \right] \]
and we have
\[ \left| \begin{array}{lr} \frac{-6 P k}{25} q_1^{-8/5} q_2^{1/5} & \frac{2 P k}{25} q_1^{-3/5}q_2^{-4/5}\\  \frac{2 P k}{25} q_1^{-3/5} q_2^{-4/5} & \frac{-4 P k}{25} q_1^{2/5}q_2^{-9/5}\\   \end{array} \right| 
=\frac{24P^2 k^2}{25^2}q_1^{-6/5}q_2^{-8/5}-\frac{4P^2 k^2}{25^2}q_1^{-6/5}q_2^{-8/5} \]
\[=\frac{20P^2 k^2}{25^2}q_1^{-6/5}q_2^{-8/5}
\] 
so the determinant is always nonzero for $p_1>0$, $p_2>0$ and the matrix is negative definite.
\item
\[
\left[
\begin{array}{lr}
0&0 \\ 
0&0\\
\end{array}
\right] =
\left[ \begin{array}{lr} \frac{-6 P k}{25} q_1^{-8/5} q_2^{1/5} & \frac{2 P k}{25} q_1^{-3/5}q_2^{-4/5}\\  \frac{2 P k}{25} q_1^{-3/5} q_2^{-4/5} & \frac{-4 P k}{25} q_1^{2/5}q_2^{-9/5}\\   \end{array} \right] 
\left[
\begin{array}{lr}
\frac{\partial q_1}{\partial p_1} & \frac{\partial q_1}{\partial p_2} \\ 
\frac{\partial q_2}{\partial p_1} & \frac{\partial q_2}{\partial p_2} \\
\end{array}
\right]
+\left[
\begin{array}{lr}
\frac{\partial g_1}{\partial p_1} & \frac{\partial g_1}{\partial p_2} \\ 
\frac{\partial g_2}{\partial p_1} & \frac{\partial g_2}{\partial p_2} \\
\end{array}
\right]
\]

\[
\left[
\begin{array}{lr}
0&0 \\ 
0&0\\
\end{array}
\right] =
\left[ \begin{array}{lr} \frac{-6 P k}{25} q_1^{-8/5} q_2^{1/5} & \frac{2 P k}{25} q_1^{-3/5}q_2^{-4/5}\\  \frac{2 P k}{25} q_1^{-3/5} q_2^{-4/5} & \frac{-4 P k}{25} q_1^{2/5}q_2^{-9/5}\\   \end{array} \right] 
\left[
\begin{array}{lr}
\frac{\partial q_1}{\partial p_1} & \frac{\partial q_1}{\partial p_2} \\ 
\frac{\partial q_2}{\partial p_1} & \frac{\partial q_2}{\partial p_2} \\
\end{array}
\right]
+\left[
\begin{array}{lr}
-1 & 0\\ 
0 &-1 \\
\end{array}
\right]
\]


\item 

\[
\left[
\begin{array}{lr}
0&0 \\ 
0&0\\
\end{array}
\right] =
\left[ \begin{array}{lr} \frac{-6 P k}{25} q_1^{-8/5} q_2^{1/5} & \frac{2 P k}{25} q_1^{-3/5}q_2^{-4/5}\\  \frac{2 P k}{25} q_1^{-3/5} q_2^{-4/5} & \frac{-4 P k}{25} q_1^{2/5}q_2^{-9/5}\\   \end{array} \right] 
\left[
\begin{array}{lr}
\frac{\partial q_1}{\partial p_1} & \frac{\partial q_1}{\partial p_2} \\ 
\frac{\partial q_2}{\partial p_1} & \frac{\partial q_2}{\partial p_2} \\
\end{array}
\right]
+\left[
\begin{array}{lr}
-1 & 0\\ 
0 &-1 \\
\end{array}
\right]
\]
\[
\left[
\begin{array}{lr}
1 & 0\\ 
0 &1 \\
\end{array}
\right]
=
\left[ \begin{array}{lr} \frac{-6 P k}{25} q_1^{-8/5} q_2^{1/5} & \frac{2 P k}{25} q_1^{-3/5}q_2^{-4/5}\\  \frac{2 P k}{25} q_1^{-3/5} q_2^{-4/5} & \frac{-4 P k}{25} q_1^{2/5}q_2^{-9/5}\\   \end{array} \right] 
\left[
\begin{array}{lr}
\frac{\partial q_1}{\partial p_1} & \frac{\partial q_1}{\partial p_2} \\ 
\frac{\partial q_2}{\partial p_1} & \frac{\partial q_2}{\partial p_2} \\
\end{array}
\right]
\]
but since the matrix on the left is just the identity we have 
\[
\left[ \begin{array}{lr} \frac{-6 P k}{25} q_1^{-8/5} q_2^{1/5} & \frac{2 P k}{25} q_1^{-3/5}q_2^{-4/5}\\  \frac{2 P k}{25} q_1^{-3/5} q_2^{-4/5} & \frac{-4 P k}{25} q_1^{2/5}q_2^{-9/5}\\   \end{array} \right] ^{-1}
=
\left[
\begin{array}{lr}
\frac{\partial q_1}{\partial p_1} & \frac{\partial q_1}{\partial p_2} \\ 
\frac{\partial q_2}{\partial p_1} & \frac{\partial q_2}{\partial p_2} \\
\end{array}
\right]
\]
\[
\left[ \begin{array}{lr} \frac{-6 P k}{25} q_1^{-8/5} q_2^{1/5} & \frac{2 P k}{25} q_1^{-3/5}q_2^{-4/5}\\  \frac{2 P k}{25} q_1^{-3/5} q_2^{-4/5} & \frac{-4 P k}{25} q_1^{2/5}q_2^{-9/5}\\   \end{array} \right] ^{-1}
=
 \frac{25}{2 P k} \left( \left[ \begin{array}{lr} -3  q_1^{-8/5} q_2^{1/5} &   q_1^{-3/5}q_2^{-4/5}\\    q_1^{-3/5} q_2^{-4/5} & -2  q_1^{2/5}q_2^{-9/5}\\   \end{array} \right] ^{-1}\right)
\]
and since
\[
\left| \begin{array}{lr} -3  q_1^{-8/5} q_2^{1/5} &   q_1^{-3/5}q_2^{-4/5}\\    q_1^{-3/5} q_2^{-4/5} & -2  q_1^{2/5}q_2^{-9/5}\\   \end{array} \right| = 5 q_1^{-6/5}q_2^{-8/5}
\]
\[
\left[ \begin{array}{lr} \frac{-6 P k}{25} q_1^{-8/5} q_2^{1/5} & \frac{2 P k}{25} q_1^{-3/5}q_2^{-4/5}\\  \frac{2 P k}{25} q_1^{-3/5} q_2^{-4/5} & \frac{-4 P k}{25} q_1^{2/5}q_2^{-9/5}\\   \end{array} \right] ^{-1}
=
 \frac{25}{2 P k} 5 q_1^{-6/5}q_2^{-8/5}  \left[ \begin{array}{lr}-2  q_1^{2/5}q_2^{-9/5}&  - q_1^{-3/5} q_2^{-4/5} \\   - q_1^{-3/5}q_2^{-4/5} &    -3  q_1^{-8/5} q_2^{1/5}  \end{array} \right] 
\]

\end{enumerate}


\end{document}
