\documentclass[12pt]{article}
\setlength\headheight{14.5pt}
\title{Homework}
\author{Frederick Robinson}
\date{16 February 2010}
\usepackage{amsfonts}
\usepackage{fancyhdr}
\usepackage{amsthm}
\pagestyle{fancyplain}

\begin{document}



\lhead{Frederick Robinson}
\rhead{Math 368: Optimization}

   \maketitle

\setcounter{tocdepth}{2} 

\tableofcontents

\section{Book Problems}

\subsection{Problem 9.3.8}
\subsubsection{Question}
\begin{enumerate}
\item A solution is called a basic solution if $m$ or fewer of the variables are nonzero
\item The basic feasible solutions correspond to the extreme points of the feasible region.
\item The bottom entry in the right column of a simplex tableau gives the maximum value of the objective function.
\end{enumerate}

\subsubsection{Answer}
\begin{enumerate}
\item \emph{True}. Page 30 states "A solution to this system is called a basic solution if no more than $m$ of the variables are nonzero" 
\item \emph{True.} Again, Page 30 says "Geometrically, these basic feasible solutions correspond to the extreme points of the feasible set. "
\item \emph{False.} This is partially true. The entry in question gives the value of the objective function at the point in question which is sometimes the maximum value, but could be less if the algorithm has not yet been run.
\end{enumerate}

\subsection{Problem 9.3.12}
\subsubsection{Question}
\[\begin{array}{lllll} 
\mathrm{Maximize} & 2x_1& +5x_2 & +3x_3\\
\mathrm{subject\ to}& x_1&+2x_2& &\leq 28\\
&2x_1&&+ 4 x_3& \leq 16\\
&&x_2&+x_3&\leq 12\\
\mathrm{and\ }x_1\geq 0, x_2\geq 0.
 \end{array}\]

\subsubsection{Answer}
In this case we have only resource contraints, so we can set up the simplex as described on page 6 of the notes.
\[\left[\begin{array}{lll|lll|l}
x_1& x_2&x_3&s_1& s_2 & s_3 & \\
\hline 
1 & 2 &&1&&&28\\
2&&4&&1&&16\\
&1&1&&&1&12\\
\hline
-2&-5&-3&&&&0
\end{array}\right]\]
Now we can just perform the algorithm with this matrix as follows
\[
\left[\begin{array}{lll|lll|l}
x_1& x_2&x_3&s_1& s_2 & s_3 & \\
\hline 
1 & 2 &0&1&0&0&28\\
2&0&4&0&1&0&16\\
0&1&1&0&0&1&12\\
\hline
-2&-5&-3&0&0&0&0
\end{array}\right]
\Rightarrow
\left[\begin{array}{lll|lll|l}
x_1& x_2&x_3&s_1& s_2 & s_3 & \\
\hline 
1 & 0 &-2&1&0&-2&4\\
2&0&4&0&1&0&16\\
0&1&1&0&0&1&12\\
\hline
-2&0&2&0&0&5&60
\end{array}\right]
\]
\[
\Rightarrow
\left[\begin{array}{lll|lll|l}
x_1& x_2&x_3&s_1& s_2 & s_3 & \\
\hline 
1 & 0 &-2&1&0&-2&4\\
0&0&8&-2&1&4&8\\
0&1&1&0&0&1&12\\
\hline
0&0&-2&2&0&1&68
\end{array}\right]
\Rightarrow
\left[\begin{array}{lll|lll|l}
x_1& x_2&x_3&s_1& s_2 & s_3 & \\
\hline 
1 & 0 &0&\frac{1}{2}&\frac{1}{4}&-1&6\\
0&0&1&-\frac{1}{4}&\frac{1}{8}&\frac{1}{2}&1\\
0&1&0&\frac{1}{4}&-\frac{1}{8}&\frac{1}{2}&11\\
\hline
0&0&0&\frac{3}{2}&\frac{1}{4}&2&70
\end{array}\right]
\]
So the solution $(6,11,1)$ Maximizes the above problem giving in particular\\$f(6,11,1)=70$. Moreover one can check that this solution does indeed satisfy all the constraints as desired.


\subsection{Problem 9.3.14}
\subsubsection{Question}
\[\begin{array}{ccccc} 
\mathrm{Minimize} & 2x_1& +3x_2 & +3x_3\\
\mathrm{subject\ to}& x_1&-2x_2& &\geq -8\\
&&2x_2&+  x_3& \geq 15\\
&2x_1&-x_2&+x_3&\leq 25\\
\mathrm{and\ }x_1\geq 0, x_2\geq 0, x_3 \geq 0.
 \end{array}\]
\subsubsection{Answer}
Since in this case there are more than just resource constraints our matrix is slightly more complex. We need to rewrite the constraint involving a negative number as the opposite constraint on a positive value. Moreover, we must invert the sign of the minimization problem to make it a maximization problem before constructing the matrix.
\[
\left[ \begin{array} {lll|lll|l|l}
x_1 & x_2 & x_3 &s_1 & s_2 &s_3 &r_2&\\
\hline
-1&2&&1&&&&8\\
&2&1&&-1&&1&15\\
2&-1&1&&&1&&25\\
\hline
-1&-3&-2&-1&1&-1&&-48\\
2 & 3 & 3 &&&&&\\
\end{array}\right]
\]
Now we just apply the algorithm
\[
\left[ \begin{array} {lll|lll|l|l}
x_1 & x_2 & x_3 &s_1 & s_2 &s_3 &r_2&\\
\hline
-1&2&0&1&0&0&0&8\\
0&2&1&0&-1&0&1&15\\
2&-1&1&0&0&1&0&25\\
\hline
-1&-3&-2&-1&1&-1&0&-48\\
2 & 3 & 3 &0&0&0&0&0\\
\end{array}\right]
\]
\[
\Rightarrow
\left[ \begin{array} {lll|lll|l|l}
x_1 & x_2 & x_3 &s_1 & s_2 &s_3 &r_2&\\
\hline
-\frac{1}{2}&1&0&\frac{1}{2}&0&0&0&4\\
1&0&1&-1&-1&0&1&7\\
\frac{3}{2}&0&1&\frac{1}{2}&0&1&0&29\\
\hline
-\frac{5}{2}&0&-2&\frac{1}{2}&1&-1&0&-36\\
\frac{1}{2} & 0 & 3 &-\frac{3}{2}&0&0&0&-12\\
\end{array}\right]
\]
\[
\Rightarrow
\left[ \begin{array} {lll|lll|l|l}
x_1 & x_2 & x_3 &s_1 & s_2 &s_3 &r_2&\\
\hline
0&1&\frac{1}{2}&0&-\frac{1}{2}&0&\frac{1}{2}&\frac{15}{2}\\
1&0&1&-1&-1&0&1&7\\
0&0&-\frac{1}{2}&4&\frac{3}{2}&1&-\frac{3}{2}&\frac{37}{2}\\
\hline
0&0&\frac{1}{2}&-2&-\frac{3}{2}&-1&\frac{5}{2}&-\frac{37}{2}\\
0 & 0 & \frac{5}{2} &-1&\frac{1}{2}&0&-\frac{1}{2}&-\frac{31}{2}\\
\end{array}\right]
\]

\[
\Rightarrow
\left[ \begin{array} {lll|lll|l|l}
x_1 & x_2 & x_3 &s_1 & s_2 &s_3 &r_2&\\
\hline
 0 & 1 & \frac{1}{2} & 0 & -\frac{1}{2} & 0 & \frac{1}{2} & \frac{15}{2} \\
 1 & 0 & \frac{3}{4} & 0 & -\frac{1}{4} & \frac{1}{2} & \frac{1}{4} & \frac{65}{4} \\
 0 & 0 & -\frac{1}{4} & 1 & \frac{3}{4} & \frac{1}{2} & -\frac{3}{4} & \frac{37}{4} \\
 \hline
 0 & 0 & 0 & 0 & 0 & 0 & 1 & 0 \\
 0 & 0 & 0 & 0 & 2 & -1 & -2 & -55
\end{array}\right]
\]

Now we may remove the artificial variables 

\[
\Rightarrow
\left[ \begin{array} {lll|lll|l}
x_1 & x_2 & x_3 &s_1 & s_2 &s_3 &\\
\hline
 0 & 1 & \frac{1}{2} & 0 & -\frac{1}{2} & 0 & \frac{15}{2} \\
 1 & 0 & \frac{3}{4} & 0 & -\frac{1}{4} & \frac{1}{2} & \frac{65}{4} \\
 0 & 0 & -\frac{1}{4} & 1 & \frac{3}{4} & \frac{1}{2} & \frac{37}{4} \\
 \hline
 0 & 0 & 0 & 0 & 2 & -1  & -55
\end{array}\right]
\]

\[
\Rightarrow
\left[ \begin{array} {lll|lll|l}
x_1 & x_2 & x_3 &s_1 & s_2 &s_3 &\\
\hline
 0 & 1 & \frac{1}{2} & 0 & -\frac{1}{2} & 0 & \frac{15}{2} \\
 1 & 0 & 1 & -1 & -1 & 0 & 7 \\
 0 & 0 & -\frac{1}{2} & 2 & \frac{3}{2} & 1 & \frac{37}{2} \\
 \hline
 0 & 0 & -\frac{1}{2} & 2 & \frac{7}{2} & 0 & -\frac{73}{2}
\end{array}\right]
\]

\[
\Rightarrow
\left[ \begin{array} {lll|lll|l}
x_1 & x_2 & x_3 &s_1 & s_2 &s_3 &\\
\hline
 -\frac{1}{2} & 1 & 0 & \frac{1}{2} & 0 & 0 & 4 \\
 1 & 0 & 1 & -1 & -1 & 0 & 7 \\
 \frac{1}{2} & 0 & 0 & \frac{3}{2} & 1 & 1 & 22 \\
 \hline
 \frac{1}{2} & 0 & 0 & \frac{3}{2} & 3 & 0 & -33
\end{array}\right]
\]



So our optimum is just $f(0,4,7)=33$.


\section{Supplemental Problems}

\subsection{Problem L.3}
\subsubsection{Question}
Solve the linear program by the Simplex Algorithm
\[\begin{array}{ccccc} 
\mathrm{Maximize} & x_1& +2x_2 & +3x_3\\
\mathrm{subject\ to}& x_1&+2x_2&+x_3 &=36\\
&2x_1&+x_2&+ 4 x_3& \geq 12\\
\mathrm{and\ }x_1\geq 0, x_2\geq 0, x_3 \geq 0.
 \end{array}\]
\subsubsection{Answer}
We begin by setting up the matrix as follows
\[
\left[ \begin{array} {lll|l|ll|l}
x_1 & x_2 & x_3  & s_2 & r_1 &r_2\\
\hline
1&2&1&&1&&36\\
2&4&1&-1&&1&12\\
\hline
-3&-6&-2&1&&&-48\\
-1&-2&-3&&&&
\end{array}\right]
\]


Now we apply the simplex algorithm


\[
\left[ \begin{array} {lll|l|ll|l}
x_1 & x_2 & x_3  & s_2 & r_1 &r_2\\
\hline
1&2&1&0&1&0&36\\
2&4&1&-1&0&1&12\\
\hline
-3&-6&-2&1&0&0&-48\\
-1&-2&-3&0&0&0&0
\end{array}\right]
\Rightarrow
\left[ \begin{array} {lll|l|ll|l}
x_1 & x_2 & x_3  & s_2 & r_1 &r_2\\
\hline
0&0&\frac{1}{2}&\frac{1}{2}&1&-\frac{1}{2}&30\\
\frac{1}{2}&1&\frac{1}{4}&-\frac{1}{4}&0&\frac{1}{4}&3\\
\hline
0&0&-\frac{1}{2}&-\frac{1}{2}&0&\frac{3}{2}&-30\\
0&0&-\frac{5}{2}&-\frac{1}{2}&0&\frac{1}{2}&6
\end{array}\right]
\]
\[
\left[ \begin{array} {lll|l|ll|l}
x_1 & x_2 & x_3  & s_2 & r_1 &r_2\\
\hline
-1&-2&0&1&1&-1&24\\
2&4&1&-1&0&1&12\\
\hline
1&2&0&-1&0&2&-24\\
5&10&0&-3&0&3&36
\end{array}\right]
\Rightarrow
\left[ \begin{array} {lll|l|ll|l}
x_1 & x_2 & x_3  & s_2 & r_1 &r_2\\
\hline
-1&-2&0&1&1&-1&24\\
1&2&1&0&1&0&36\\
\hline
0&0&0&0&1&1&0\\
2&4&0&0&3&0&108
\end{array}\right]
\]
Now we may drop the artificial variables since they are no longer pivots and are zero
\[
\left[ \begin{array} {lll|l|l}
x_1 & x_2 & x_3  & s_2 \\
\hline
-1&-2&0&1&24\\
1&2&1&0&36\\
\hline
2&4&0&0&108
\end{array}\right]
\Rightarrow
\left[ \begin{array} {lll|l|l}
x_1 & x_2 & x_3  & s_2 \\
\hline
0&0&0&1&\\
0&0&1&0&36\\
\hline
2&4&0&0&108
\end{array}\right]
\]
So we see that the solution to our problem is given by $f(0,0,36)=108$




\end{document}
